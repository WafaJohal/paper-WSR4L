\documentclass{sig-alternate-05-2015}
\usepackage[utf8]{inputenc}

\begin{document}

\CopyrightYear{2017}
\setcopyright{rightsretained}
\conferenceinfo{HRI '17 Companion}{March 06-09, 2017, Vienna, Austria}
\isbn{978-1-4503-4885-0/17/03}
\doi{http://dx.doi.org/10.1145/3029798.3029801}

%\clubpenalty=10000
%\widowpenalty = 10000---

\title{Robots for Learning}

\numberofauthors{10} 
\author{
\alignauthor
Wafa Johal\\
       \affaddr{CHILI/LSRO Labs}\\
       \affaddr{{\'E}cole Polytechnique}\\
       \affaddr{F{\'e}d{\'e}rale Lausanne}\\
       \affaddr{Lausanne, Switzerland}\\
       \email{wafa.johal@epfl.ch}
% 2nd. author
\alignauthor
Paul Vogt\\
       \affaddr{Tilburg center for Cognition and Computation}\\
       \affaddr{Tilburg University}\\
       \affaddr{Netherlands}\\
       \email{p.a.vogt@uvt.nl}
% 3rd. author
\alignauthor James Kennedy\\
       \affaddr{Centre for Robotics and Neural Systems}\\
       \affaddr{Plymouth University}\\
       \affaddr{United Kingdom}\\
       \email{james.kennedy\\@plymouth.ac.uk}
\and  % use '\and' if you need 'another row' of author names
% 4th. author
\alignauthor Mirjam de Haas\\
	   \affaddr{Tilburg center for Cognition}\\
	   \affaddr{and Computation}\\
	   \affaddr{Tilburg University}\\
	   \affaddr{Netherlands}\\
	   \email{mirjam.dehaas@uvt.nll}
% 5th. author
\alignauthor Ana Paiva\\
       \affaddr{Instituto Superior T{\'e}cnico}\\
       \affaddr{University of Lisbon}\\
       \affaddr{Portugal}\\
       \email{ana.paiva@inesc-id.pt}
% 6th. author
\alignauthor Ginevra Castellano\\
	   \affaddr{Department of Information Technology}\\
       \affaddr{Uppsala University}\\
       \affaddr{Sweden}\\
       \email{ginevra.castellano\\@it.uu.se}
}
% There's nothing stopping you putting the seventh, eighth, etc.
% author on the opening page (as the 'third row') but we ask,
% for aesthetic reasons that you place these 'additional authors'
% in the \additional authors block, viz.
\additionalauthors{Additional authors: 
Sandra Okita (Teachers College - Columbia University, United States, email: {\texttt{okita@tc.colum-bia.edu}}), Fumihide Tanaka
(University of Tsukuba, Japan, email: {\texttt{tanaka@iit.tsukuba.ac.jp}}), 
Tony Belpaeme (Centre for Robotics and Neural Systems, Plymouth University, U.K. and Ghent University, Belgium, email: {\texttt{tony.belpa-eme@plymouth.ac.uk}}) and
Pierre Dillenbourg (CHILI Lab, {\'E}cole F{\'e}d{\'e}rale Polytechnique Lausanne, Switzerland, email: {\texttt{pierre.dillenbourg@epfl.ch}}). 
}
%\date{30 July 1999}
% Just remember to make sure that the TOTAL number of authors
% is the number that will appear on the first page PLUS the
% number that will appear in the \additionalauthors section.

%TODO fix author emails going outside page boundary

\maketitle

%-----------------------------------------------------------------------------
\begin{abstract}
An increasing amount of Human-Robot Interaction (HRI) research is focused on the 
development of social robot tutors. While robots have been popular as a tool for 
STEM teaching, the use of robots as tutors is novel. The field of HRI has 
started to report on how to make effective robot tutors. However, many 
challenges remain. For instance, what interaction strategies aid learning, and 
which hamper learning? How can we deal with the current technical limitations of 
robots? Answering these and other questions requires a multidisciplinary effort, 
including contributions from pedagogy, developmental psychology, (computational) 
linguistics, artificial intelligence and HRI, among others. This abstract 
provides an overview of the current state-of-the-art in robot tutors and 
describes the aims of the Robots for Learning (R4L) workshop in bringing 
together a multidisciplinary audience for furthering the development of 
market-ready educational robots.
\end{abstract}
%-----------------------------------------------------------------------------

%TODO JK: do we need to include this? It wasn't required on the paper for full papers so removed for now
%
% The code below should be generated by the tool at
% http://dl.acm.org/ccs.cfm
% Please copy and paste the code instead of the example below. 
%
\begin{CCSXML}
	<ccs2012>
	<concept>
	<concept_id>10010520.10010553.10010554.10010558</concept_id>
	<concept_desc>Computer systems organization~External interfaces for robotics</concept_desc>
	<concept_significance>500</concept_significance>
	</concept>
	<concept>
	<concept_id>10010405.10010489</concept_id>
	<concept_desc>Applied computing~Education</concept_desc>
	<concept_significance>500</concept_significance>
	</concept>
	<concept>
	<concept_id>10003120.10003121</concept_id>
	<concept_desc>Human-centered computing~Human computer interaction (HCI)</concept_desc>
	<concept_significance>500</concept_significance>
	</concept>
	</ccs2012>
\end{CCSXML}

\ccsdesc[500]{Computer systems organization~External interfaces for robotics}
\ccsdesc[500]{Applied computing~Education}
\ccsdesc[100]{Human-centered computing~Human computer interaction (HCI)}

%
% End generated code
%

%
%  Use this command to print the description
%
%\printccsdesc

\keywords{Human-Robot Interaction, Robots in Education, Tutor Robots, Child-Robot Interaction}

%-----------------------------------------------------------------------------
\section{Introduction}
%-----------------------------------------------------------------------------
% 1 paragraph high-level overview
%TODO refs throughout this par
An increasing amount of HRI research is focused on the development of social 
robots acting as tutors. While robots have been popular as a focus for STEM 
teaching (see Lego Mindstorms or Thymio \cite{riedo2012two}), the use of robots 
as tutors is novel. The field of HRI has started reporting on how to make 
effective robot tutors and how to measure their efficacy 
\cite{kennedy2016social,tanaka2015pepper}. Social robot tutors have the 
potential to enhance learning via kinesthetic interaction \cite{}, can improve 
the learner's self-esteem \cite{}, and can provide empathic feedback 
\cite{castellano2013towards}. Finally, robots have been shown to engage the 
learner, to motivate her in the learning task or to enhance collaboration in a 
group \cite{}. However, many challenges remain and this workshop aims to bring 
together a multidisciplinary group of researchers to discuss these challenges 
and share expertise.

% 1 paragraph talk about previous version of workshop (what was done there, and what this will add)
The second iteration of this workshop builds on the previous version hosted at 
the IEEE International Symposium on Robot and Human Interactive Communication 
(RO-MAN), 2016. The previous workshop utilised keynote speakers, participant 
speakers, and small group discussions to raise issues and challenges facing the 
community researching robots for use in delivering educational content. The 
second version of this workshop seeks to engage with more researchers in the 
field, and draw a more multidisciplinary audience to further the development of 
market-ready educational robots.

%-----------------------------------------------------------------------------
\section{Background}
%-----------------------------------------------------------------------------
% overview state-of-the-art for robots for learning (maybe also clarify what is meant by the term)

% close by highlighting some challenges (which we can talk about addressing in outline...)
%However, many questions still remain. For instance, what interaction strategies aid learning, and which hamper learning? How can we deal with the current technical limitations of robots? How should effective lessons be developed and implemented on a robot? Answering these and other questions requires a multidisciplinary effort, including contributions from pedagogy, developmental psychology, (computational) linguistics, artificial intelligence and HRI, among others. 

%-----------------------------------------------------------------------------
\section{Outline of the Workshop}
%-----------------------------------------------------------------------------
% who is the workshop intended for?
The aim of this workshop is to engage scholars who wish to gain expertise in 
education and in robotics. Participants will benefit from hearing from the 
forefront of field and from discussions on how to move from fundamental research 
towards the development of market-ready educational robots.

% how will the aims be achieved on the day?
The workshop aims will be achieved through

%-------------------------------------------
%\section{Organizers}
% some workshop abstracts include details about the organizers...but maybe we should avoid that given that it would produce a short novel.

%-----------------------------------------------------------------------------
\section{Acknowledgments}
We would like to thank the Swiss National Science Foundation 
href{http://www.nccr-robotics.ch/}{National Centre of Competence in Research 
Robotics}, the EU H2020 L2TOR project (grant no. 688014),\dots

%-----------------------------------------------------------------------------
\bibliographystyle{abbrv}
\bibliography{hriwk10_r4l}  

\end{document}
